\documentclass[12pt]{article}

\usepackage{CJKutf8}
% 打中文時請務必加這個Package

\usepackage{epsfig,amsmath,amssymb,latexsym}
\usepackage{graphicx,epsfig,color,epsf,psfrag,hhline,amsmath,amssymb,textcomp}
%\usepackage{hyperref}
\usepackage{enumitem}

\setlength {\topmargin}{-1.0in}
\setlength {\textheight}{10in}
\setlength {\oddsidemargin}{-0.25in}
\setlength {\evensidemargin}{-0.25in}
\setlength {\textwidth}{6.75in}
\setlength {\parskip}{8pt plus 2pt minus 1pt}

\newtheorem{theorem}{Theorem}[section]
\newtheorem{definition}{Definition}[section]
\newtheorem{lemma}{Lemma}[section]

\newcommand{\comb}[2]{\left (\begin{array}{c} #1 \\ #2 \end{array} \right )}

\newcommand{\inlinecomb}[2]{\mbox{\scriptsize $\left (\begin{array}{c} #1 \\ #2 \end{array} \right )$\normalsize}}

\newcommand {\bsolution}{\noindent {\em Solution:} \ }

\newcommand{\esolution}{\hfill $\Diamond$ \\ \vspace{.3cm}}

%************************** Figure**********************************
\newcommand {\bfig}[2] {\begin{figure}[htbp]
                        \centerline {
                         \epsfig{figure={#1},clip=,width={#2}}}}

\newcommand {\efig}[2]{ \caption{#2}
                        \label{fig:#1}
                        \end{figure}
                        \mymarginpar{fig:#1}}
\newcommand {\rfig}[1]{Figure \ref{fig:#1}}

%%%%%%%%%%%%%%%%%%%%%%%%%%%%%%%%%%%%%%%%%%%%%%%%%

\begin{document}
\thispagestyle{empty}
\begin{center}
{\Large \noindent COM 530500 {\bf Network Science Final Project} \\
\large {{\sc Due:} Thursday, January 20, 2022}  \\
}
\emph{No late homework will be accepted}.
\end{center}


\noindent {\begin{CJK}{UTF8}{bsmi}
{\bf 班級:} 通訊所
\end{CJK}}

\noindent  {\begin{CJK}{UTF8}{bsmi}
{\bf 姓名:} 王小明
\end{CJK}}

\noindent {\begin{CJK}{UTF8}{bsmi}
{\bf 學號:} 105061199
\end{CJK}}

\bigskip


%---------------------------------------------------------------------------
% Problem 1
%---------------------------------------------------------------------------

\noindent {\bf Problem 1. (40\%)} Consider the {\bf ego-Facebook} \cite{leskovec2012learning} dataset. A node in this dataset represents a user on Facebook, and an edge between two nodes represents the relationship between two users.
\begin{enumerate}[label=(\alph*)]
	\item (10\%) List some statistical information of this dataset, such as the number of nodes, number of edges, average clustering coefficient, diameter, average degree, maximum degree, etc.
	\item (10\%) Visualize the dataset by plotting it.
	\item (10\%) Plot the degree distribution with log-log scale.
	\item (10\%) List the top 10 nodes ranked by the following centrality measures.
	\begin{itemize}
		\item Degree centrality
		\item Katz centrality
		\item Eigenvector centrality
		\item Betweenness centrality
		\item Closeness centrality
	\end{itemize}
\end{enumerate}

\bsolution
%---------------------------------------------------------------------------
Type your answer here.
%---------------------------------------------------------------------------
\esolution


%---------------------------------------------------------------------------
% Problem 2
%---------------------------------------------------------------------------

\noindent {\bf Problem 2. (60\%+bonus 10\%)} In this problem, we want to investigate the disease propagation by the independent cascade (IC) model in {\bf ego-Facebook} \cite{leskovec2012learning} dataset. Assume the propagation probability is $\phi$, and the set of seeds nodes $S$ are randomly selected. Collect the set of infected nodes within the distance $D$ of the seed nodes, and calculate the prevalence rate $r_1$ (which is defined by the ratio of the number of infected nodes to the total number of nodes). Set $\phi=0.1, \vert S \vert=5$, and $D$ the diameter of the graph.

\begin{enumerate}[label=(\alph*)]
	\item (40\%) Simulate the disease propagation by IC model after removing the top $0\%$, $10\%$, $20\%$, $\ldots$, $50\%$ of nodes from the following centrality measures respectively, and calculate the corresponding prevalence rate $r_1$. Please plot the curves of $r_1$ vs. the percentage of nodes removed. ({\it Note: Please run the simulation 100 times and average the results.})
	\begin{itemize}
		\item Degree centrality
		\item Katz centrality
		\item Eigenvector centrality
		\item Betweenness centrality
		\item Closeness centrality
	\end{itemize}
	\item (bonus 10\%) Could you find a centrality measure that achieves a better performance?
	\item (20\%) Write a report to compare and discuss the results of different centrality measures.  
\end{enumerate}

\bsolution
%---------------------------------------------------------------------------
Type your answer here.
%---------------------------------------------------------------------------
\esolution

%---------------------------------------------------------------------------
% The end of problems.
%---------------------------------------------------------------------------

%---------------------------------------------------------------------------
\iffalse
\noindent {\bf Given}: adjacency matrix $A$, transition probability $\phi$, a set of seed nodes $S$, and distance $D$.

\noindent {\bf Step 1}: Obtain $A^\prime$ by removing edges of $A$, where each edge is removed with probability $1-\phi$.

\noindent {\bf Step *}: {\bf Obtain $\tilde{A}$ by removing some specific nodes and all their edges from $A^\prime$.}

\noindent {\bf Step 2}: Define the $n \times 1$ seed vector $x$ by $x_i=1$ for $i \in S$ and $x_i=0$ for $i \notin S$.

\noindent {\bf Step 3}: Calculate $y=(\tilde{A}+I)^Dx$ with the AND and the OR operations.

\noindent {\bf Step 4}: Obtain the number of infected nodes by counting the number of 1's in vector $y$.

\noindent {\bf Step 5}: Repeat {\bf Step 1} to {\bf Step 5} for $100$ times, and average the results.

\noindent {\bf Step 6}: Calculate the prevalence rate $r_1$.
\fi
%---------------------------------------------------------------------------

%---------------------------------------------------------------------------
% References
%---------------------------------------------------------------------------
\begin{thebibliography}{9}
\bibitem{leskovec2012learning}
J.~Leskovec and J.~J. Mcauley, ``Learning to discover social circles in ego networks,'' in \emph{Advances in neural information processing systems},
2012, pp. 539--547.

\end{thebibliography}

\end{document}
