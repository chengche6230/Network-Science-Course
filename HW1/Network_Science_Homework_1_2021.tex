\documentclass[12pt]{article}

\usepackage{CJKutf8}
% 打中文時請務必加這個Package

\usepackage{epsfig,amsmath,amssymb,latexsym}
\usepackage{graphicx,epsfig,color,epsf,psfrag,hhline,amsmath,amssymb,textcomp}
\usepackage{hyperref}
\usepackage{enumitem}

\setlength {\topmargin}{-1.0in}
\setlength {\textheight}{10in}
\setlength {\oddsidemargin}{-0.25in}
\setlength {\evensidemargin}{-0.25in}
\setlength {\textwidth}{6.75in}
\setlength {\parskip}{8pt plus 2pt minus 1pt}

\newtheorem{theorem}{Theorem}[section]
\newtheorem{definition}{Definition}[section]
\newtheorem{lemma}{Lemma}[section]

\newcommand{\comb}[2]{\left (\begin{array}{c} #1 \\ #2 \end{array} \right )}

\newcommand{\inlinecomb}[2]{\mbox{\scriptsize $\left (\begin{array}{c} #1 \\ #2 \end{array} \right )$\normalsize}}

\newcommand {\bsolution}{\noindent {\em Solution:} \ }

\newcommand{\esolution}{\hfill $\Diamond$ \\ \vspace{.3cm}}

%************************** Figure**********************************
\newcommand {\bfig}[2] {\begin{figure}[htbp]
                        \centerline {
                         \epsfig{figure={#1},clip=,width={#2}}}}

\newcommand {\efig}[2]{ \caption{#2}
                        \label{fig:#1}
                        \end{figure}
                        \mymarginpar{fig:#1}}
\newcommand {\rfig}[1]{Figure \ref{fig:#1}}

%%%%%%%%%%%%%%%%%%%%%%%%%%%%%%%%%%%%%%%%%%%%%%%%%

\begin{document}
\thispagestyle{empty}
\begin{center}
{\Large \noindent COM 530500 {\bf Network Science Homework \#1} \\
\large {{\sc Due:} Thursday, October 14, 2021}  \\
}
\emph{No late homework will be accepted}.
\end{center}


\noindent {\begin{CJK}{UTF8}{bsmi}
{\bf 班級:} 資應所
\end{CJK}}

\noindent  {\begin{CJK}{UTF8}{bsmi}
{\bf 姓名:} 鄭程哲
\end{CJK}}

\noindent {\begin{CJK}{UTF8}{bsmi}
{\bf 學號:} 110065512
\end{CJK}}

\bigskip

%---------------------------------------------------------------------------
% Problem 1
%---------------------------------------------------------------------------
\begin{figure}[h]
	\centering
	\includegraphics[width=0.4\textwidth]{NS_Hw1_a.jpg}
	\caption{Network (a).}
	\label{NS_Hw1_a}
\end{figure}

\noindent {\bf Problem 1.\bf(10\%)}
\begin{enumerate}[label=(\alph*)]
	\item (5\%) Write down the adjacency matrix of network (a).
	\item (5\%) Write down the cocitation matrix of network (a).
\end{enumerate}

\bsolution
%---------------------------------------------------------------------------

(a) $\begin{bmatrix}
	0 & 1 & 0 & 0 & 1 \\
	0 & 0 & 1 & 0 & 0 \\
	1 & 0 & 0 & 0 & 1 \\
	0 & 1 & 1 & 0 & 0 \\
	0 & 0 & 0 & 0 & 0 
	\end{bmatrix}$

(b) Cocitation matrix $C = AA^{T} = \begin{bmatrix}
	2 & 0 & 1 & 1 & 0 \\
	0 & 1 & 0 & 1 & 0 \\
	1 & 0 & 2 & 0 & 0 \\
	1 & 1 & 0 & 2 & 0 \\
	0 & 0 & 0 & 0 & 0
\end{bmatrix}$
%---------------------------------------------------------------------------
\esolution

%---------------------------------------------------------------------------
% Problem 2
%---------------------------------------------------------------------------
\begin{figure}[h]
	\centering
	\includegraphics[width=0.4\textwidth]{NS_Hw1_b.jpg}
	\caption{Network (b).}
	\label{NS_Hw1_b}
\end{figure}

\noindent {\bf Problem 2.\bf(10\%)}
\begin{enumerate}[label=(\alph*)]
	\item (5\%) Write down the incidence matrix of network (b).
	\item (5\%) Write down the projection matrix for the projection of network (b) onto its black vertices.
\end{enumerate}

\bsolution
%---------------------------------------------------------------------------

(a) I regard white dots as different groups and black dots as vertexs.\\
	Incidence matrix = $B = \begin{bmatrix}
	1 & 0 & 1 & 0 & 0 \\
	0 & 1 & 1 & 0 & 0 \\
	0 & 0 & 0 & 1 & 0 \\
	0 & 1 & 1 & 1 & 1 
\end{bmatrix}$


(b) Projection matrix $P = B^{T}B =\begin{bmatrix}
	1 & 0 & 1 & 0 & 0 \\
	0 & 2 & 2 & 1 & 1 \\
	1 & 2 & 3 & 1 & 1 \\
	0 & 1 & 1 & 2 & 1 \\
	0 & 1 & 1 & 1 & 1
	\end{bmatrix} $

%---------------------------------------------------------------------------
\esolution


%---------------------------------------------------------------------------
% Problem 3
%---------------------------------------------------------------------------

\noindent {\bf Problem 3.\bf(10\%)}
Consider a bipartite network, with its two types of vertices. Suppose there are $n_1$ vertices of type $1$ and $n_2$ vertices of type $2$. Show that the mean degrees $c_1$ and $c_2$ of the two types are related by $c_2=\frac{n_1}{n_2}c_1$.

\bsolution
%---------------------------------------------------------------------------
If there are $m$ edges between two types of vertices, The mean degree of type 1 vertices\\[0.6em]
$c_1=\frac{m}{n_1}$, and $c_2=\frac{m}{n_2}$. Therefore, 
$\frac{c_1}{c_2}=\frac{\frac{m}{n_1}}{\frac
	{m}{n_2}}=\frac{n_2}{n_1}$, and we get 
$c_2=\frac{n_1}{n_2}c_1$.
%---------------------------------------------------------------------------
\esolution

%---------------------------------------------------------------------------
% Problem 4
%---------------------------------------------------------------------------
\noindent {\bf Problem 4.\bf(20\%)}
Given 
	\[A=
		\begin{pmatrix}
			0 & 2 & -1 \\
			2 & 3 & -2 \\
			-1 & -2 & 0
		\end{pmatrix},
	\]

\begin{enumerate}[label=(\alph*)]
	\item (10\%) Find all eigenvalues of matrix $A$.
	\item (10\%) Find an orthogonal matrix $U$ that diagonalizes $A$.
\end{enumerate}

\bsolution
%---------------------------------------------------------------------------
Type your answer here.
%---------------------------------------------------------------------------
\esolution



\newpage
%---------------------------------------------------------------------------
% Problem 5
%---------------------------------------------------------------------------
\noindent {\bf Problem 5.\bf(15\%)}
Read the tutorial from \href{https://github.com/PingEnLu/Network-Science-COM530500/tree/master/Network_Science_Python_iGraph_Tutorial}{Ping-En Lu's GitHub repository} to install Python3 and python-igraph (if you need).

Paste your screenshots of “Hello, World!” of both Python 3 (5\%) and python-igraph (5\%), and write a brief report (5\%). (For example, you can write down some problems you encountered, and how you solved them.)

\bsolution
%---------------------------------------------------------------------------
Type your answer here.
%---------------------------------------------------------------------------
\esolution

%---------------------------------------------------------------------------
% Problem 6
%---------------------------------------------------------------------------
\noindent {\bf Problem 6.\bf(35\%)}
Please download the {\bf tvshow} dataset from \href{https://github.com/PingEnLu/Network-Science-COM530500/tree/master/Network_Science_Python_iGraph_Tutorial}{Ping-En Lu's GitHub repository}, and find the following information from this dataset.

\begin{itemize}
	\item Number of nodes. (5\%)
	\item Number of edges. (5\%)
	\item Mean degree. (5\%)
	\item Maximum degree. (5\%)
	\item Diameter. (5\%)
\end{itemize}
You need to upload your {\bf python source code} to iLMS, and {\bf write a brief report (10\%) including screenshots, README file, and descriptions of your code} below the solution area. There will be no points for this problem if you do not upload your python source code to iLMS.

\bsolution
%---------------------------------------------------------------------------
Type your answer here.
%---------------------------------------------------------------------------
\esolution

%---------------------------------------------------------------------------
% The end of problems.
%---------------------------------------------------------------------------


\end{document}
